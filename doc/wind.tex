%%% File: wind.tex

\chapter{Wind Interpolation}

%----------------------------------------------------------------------

\section{Introduction}

The wind data is given in a mesh and, possibly, in some set of random
points. We have to be able to restore the proper wind vector for any
ice element on a sphere. We also would like to have a few analytical
testing fields that we can use. This chapter describes all that
related to the wind.

%----------------------------------------------------------------------

\section{Analytical Testing Fields}

First field to use is taken from~\cite{bib:fuselier2009stability},
field~1, pp.~3233--3234.

%----------------------------------------------------------------------

\section{NCEP/NCAR Reanalysis grids}

Most part of testing and generic [?] wind interpolations is based on NCEP/NCAR Reanalysis data grids. 

The data is represented as a values distributed upon the regular grid in latitude-longitude coordinates. 
  
\subsection{Basic interpolation method}

As the grid is guaranteed to be regular and the wind field itself is pretty smoth - basic interpolation is made as if it is simple Cartesian coordinate system. 
The simpliest method to obtain the interpolated value in two dimensional space is bilinear interpolation.
This method is based on the assumption that the value should smothly change along the distance between two known values. 
Saying 'smothly' means that the interpolated value should be calculated with the heuristic function, which defines the proportion law.
The methed consists of next steps:
\begin{enumerate}
 \item Constructing regular grid with all needed values in the nodes.
 \item Find the grid cell, that matches to interpolation value position.
 \item With heuristic proportion - calculating two additional values, tha are located on the same lattitude with interpolated value, but at the same time on cell` right and left boarders..
 \item With that very heuristic propotion - calculating the value we were searching for.
\end{enumerate}

This procedure means that proportion function shall be called three times in a raw. More then that - in most cases the result that is based on 'same latitude' 
additionel boarder values will differ from the result based on 'same longitude' additional boarder values.

\subsection{Proportion functions}

All circumstances described in previous section mean that proportion function should be as simple as possible, but at the same time it should preserve such field attributes 
as curl and divergence.

For satisfying such various conditions we provide several functions, that bestly suit for differents cases.

\begin{itemize}
 \item Generic proportion and interpoation. 
 
 The wind vector is represented as range-azimuth pair. Both values are considered to change linearily.
 \begin{equation}
  |\pnt{V_t}| = (1 - t)|\pnt{V_1}| + t|\pnt{V_2}|.
 \end{equation}
 \begin{equation}
   \phi _t = (1 - t)\phi_1 + t\phi_2.
 \end{equation}
 
 Where $|\pnt{V_t}|$ is interpolated velocity vector, $\phi$ is it`s azimuth, t is scalar value $( t\in[0, 1] )$
 
 \item Simple proportion and interpolation.
 
 The wind vector is represented as east-north components pair. Both values are considered to change linearily.
 \begin{equation}
  \pnt{V_t} = (1 - t)\pnt{V_1} + t\pnt{V_2}.
 \end{equation}
 
 \item SLERP proportion and interpolation.
 
 According to \url{https://en.wikipedia.org/wiki/Slerp}.
 \begin{equation}
  \pnt{V_t} = \frac{\sin[(1 - t)\Omega]}{\sin\Omega}\pnt{V_1} + \frac{\sin[t\Omega]}{\sin\Omega}\pnt{V_2}.
 \end{equation}
 
 Where $\Omega$ is the angle between $\pnt{V_1}$ and $\pnt{V_2}$
 
\end{itemize}

%----------------------------------------------------------------------

\section{Advanced interpolation (expected soon)}

For more accurate and meaningfull interpolation several more complicated methods are going to be provided.

The main aim is to take into account such parameters as wind`s first (and maybe second) derivative, that could be calculadet in the nodes, 
and additional random distributed wind data from various other sources.