%%% File: dynamics.tex

\chapter{Dynamics}

%----------------------------------------------------------------------

\section{Introduction}

The main problem in dynamics is to calculate 3D ``pendulum'' angular
velocity $\vct{\Omega}$ in ice-element reference frame. This vector
can be calculated exactly using 3D dynamics of a rigid body with
constraints. However, it is not necessary. A much simpler way should
be sufficient if we suppose that the ice-element size is much smaller
than the radius of the Earth. This is equivalent to the assumption
that the ice element is flat.

In spite of this assumptions we are not solving 2D cartesian
problem. All we are saying is that we are considering the dynamics on
a sphere with locally flat ice elements.

%----------------------------------------------------------------------

\section{Dynamics in ice-element reference frame}

As ice-element reference frame is a standard cartesian coordinates and
we assume that the ice-element is a flat 2D ridig body, its velocity
and rotation rate can be calculated simply by the following equations
\begin{equation}\label{eq:dynamicsset}
  \dot{V_x} = F_x/m, \qquad \dot{V_y} = F_y/m, 
  \qquad \dot\Omega_z = N/I,
\end{equation}
where $(F_x,F_y)$ are external forces that include all driving forces,
integral forces from interaction with other ice-elements, and Earth
Coriolis force, $m$ is the total mass of the ice element, $I$ is a
moment of inertia around $Z$ axis, and $N$ is a total torque
calculated in the center of mass $\vct{C}$ that includes a torque from
interaction with other ice elements and possible torque that can be
calculated from the integral $(\nabla\times F)_z$ of the driving
forces over the ice element (TODO: discuss with Jenny).

%----------------------------------------------------------------------

\section{Restoring 3D angular velocity}

It is pretty clear that if we know $V_x$, $V_y$, and $\Omega_z$, we
can restore $\vct{\Omega}$ as follows: 
\begin{equation}
  \vct{\Omega}= (-V_y/R, V_x/R, \Omega_z)^T.
\end{equation}
Indeed, $V_x$ is directed normal to $Z$ and tends to rotate the
``pendulum'' around $Y$ axis with $V_x/R$ rate. The rotation
direction is ``right'' and that is exactly $\Omega_y$. Same
considerations show that $|\Omega_x| = V_y/R$ and we need to choose
``-'' as a sign to take into account the right direction.

Taking this into account, we can rewrite the dynamics ecuations for
global angular velocity vector in Local frame:
\begin{equation}\label{eq:dynamicsH}
  \dot{\vct{\Omega}} = \vct{H} = \left( -\frac{F_y}{mR}, \frac{F_x}{mR},
  \frac{N}{I} \right)^T.
\end{equation}

%----------------------------------------------------------------------

\section{Dynamics Integration}

A straightforward 2nd order in time scheme is applicable for
Eq.~\eqref{eq:dynamicsH}:
\begin{equation}
  \left\{
  \begin{aligned}
  \Omega_x^{n+1/2} &= \Omega_x^{n-1/2} - \Delta t\,F_y^n/mR,\\
  \Omega_y^{n+1/2} &= \Omega_y^{n-1/2} + \Delta t\,F_x^n/mR,\\
  \Omega_z^{n+1/2} &= \Omega_z^{n-1/2} + \Delta t\,N^n/I\\
  \end{aligned}
  \right.
\end{equation}
Or in vector form:
\begin{equation}
  \vct{\Omega}^{n+1/2} = \vct{\Omega}^{n-1/2} + \Delta t\,\vct{H}^n.
\end{equation}


Thus, the numerical scheme for integration of motion of the ice
element is completed in terms of rotational quaternions.

