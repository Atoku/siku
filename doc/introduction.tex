% FILE: introduction.tex

\chapter{Introduction}

%---------------------------------------------------------------------
\section{Basic Information}

Siku is written on a subset of C++ language. Please, see
Sec.~\ref{sec:programing-policy} for strict programming policy.

%---------------------------------------------------------------------
\section{Software Requirements}

To compile and use the model, the following packages are required:
\begin{enumerate}
\item Python 3.0 or higher
  \begin{enumerate}
  \item \texttt{mathutils} from blender project. 
    See \url{https://code.google.com/p/blender-mathutils/}.
  \item \texttt{netcdf4-python} (recommended if you need import
    NetCDF4 forcing files).
  \item \texttt{numpy}
  \item \texttt{shapefile} (known also as \texttt{pyshp}).
  \end{enumerate}
\item HDF5 version 1.8 or higher.
\item NetCDF4 version 4.1.1 or higher.
\item GSL version 1.16 or higher.
\item Boost library version 1.53 or higher.
\item OpenGL Mathematics (GLM) version 0.9.6.1 or higher.
\item \LaTeX\/ for creating documentation.
\end{enumerate}

The packages that are necessary to compile the core are checked during
the configuration step. Python libraries are not checked.

%---------------------------------------------------------------------
\section{Programming Policy}\label{sec:programing-policy}

A strict programming policy is enforced to keep the code clean and
rational. This policy is strictly enforced in particular due to usage
of such an obscure, overloaded and heavy language as C++.

%----------------------------------------------------------------------
\section{Development Policy}


