%FILE: formulas.tex

\chapter{Formulas Summary}

\paragraph{Dynamics}

\begin{equation}\label{eq:dynamicsH-x}
  \dot{\vct{\Omega}} = \vct{H} = \left( -\frac{F_y}{mR}, \frac{F_x}{mR},
  \frac{N}{I} \right)^T.
\end{equation}
\begin{equation}
  \vct{\Omega}^{n+1/2} = \vct{\Omega}^{n-1/2} + \Delta t\,\vct{H}^n.
\end{equation}

\paragraph{Placement}

\begin{equation}\label{eq:positionomega-x}
  \untqtrn{\dot q} = \frac{1}{2}\untqtrn{q}\circ\vct{\Omega},
\end{equation}
\begin{equation}
  \untqtrn{p} = \left(1 - \frac{\Omega^2\,\Delta t}{16} + \frac{\Delta
    t}{2}\,\vct{\Omega} \right) / \left( 1 +\frac{ \Omega^2\,\Delta
    t^2}{16}\right),
\end{equation}
\begin{equation}
  \untqtrn{q}^{n+1} = \untqtrn{q}^{n} \circ \untqtrn{p}.
\end{equation}


\chapter{Formulas on July 6, 2016 (C++ Core)}

In this appendix all values marked with \underline{underlined} font 
are being stored in coordinates On Unit Sphere (with respective scale).
All the other values are in SI.
\\
The values marked with ``quotes'' are global constants, those are set
up in python scenario.
\\
Denotions:

R - planet radius in meters, $e_i$ - one of the ice elements.

\section{Elements` state update}

Elements` state is being updated in module ``mproperties''.
This update includes calculation of elements mass (according to 
melting/freezing) and moment of inertia, land fastency conditions check, 
minor error checks (like zero-thickness), e.t.c.
\\
\\
Two dimensional density (per unit of surface):
\begin{equation}
 \rho_{2d} = \sum_{i}^{10} \cdot \rho_i \cdot h_i \cdot h_{e_j,i}
\end{equation}
where $h_{e_j,i}$ is !UNDONE!
\\
Element`s mass:
\begin{equation}
 m_{e_j} = \underline{A_{e_j}} \cdot R^2 \cdot \rho_{2d}
\end{equation}
\\
Moment of inertia:
\begin{equation}
 I_{e_j} = m_{e_j} \cdot \underline{i_{e_j}} \cdot R^2
\end{equation}
\\
Element $e_j$ becomes fastened  $if$ ( $A_{overlap,e_j} > A_{min,e_j} \cdot ``fastency''$ ),
\\ where $A_{overlap,e_j}$ is total 
area of overlap of $e_j$ with shore elements and $A_{min,e_j}$ is an area
of the smallest of nearby elements (including $e_j$).

\section{Distributed forces}

All the distributed forces like wind drag, water currents, Coriolis force
are being calculated in ``forces\_mass'' module.
\\
Denotions:

$\vec{\tilde{V}}$ -interpolated wind velocity, 
$\vec{\tilde{W}}$ -interpolated water velocity.
\\
Wind and water drag forces:
\begin{equation}
 \vec{F}_{wind} = |\tilde{V}|\cdot\vec{\tilde{V}} \cdot \underline{A_{e_j}} \cdot R^2 \cdot windage_{e_j} \cdot ``windage''
\end{equation}
\begin{equation}
 \vec{F}_{water} = |\vec{\tilde{W}} - \vec{V_{e_j}}|\cdot(\vec{\tilde{W}} - \vec{V_{ e_j}}) \cdot \underline{A_{e_j}}
 \cdot R^2 \cdot anchority_{ e_j} \cdot ``anchority''
\end{equation}
\\
where ``windage'' and ``anchority'' are the general factors of interaction of
floes with air and water, $windage_{e_j}$ and $anchority_{ e_j}$ are respective
formfactors for each floe (depending on shape, density, e.t.c.).

\section{Elements interaction}

The interaction of elements is represented as a set of pairwise forces, 
those are selected with respect to each contact type (e.g. frozen joint, collision...).
Pairwise interactions are implemented in ``contact\_forces'' module.
In this section indexes '1' and '2' denote respectively two interacting 
ice elements for each contact. 

The process is devided into three steps:\\
1) generating collision data\\
2) calculating the force\\
3) applying and updating the state of the contact

\subsection{Contact data}

Contact data (denoted as $cd$) is a structure that is being generated
at every step for each particular contact (the pair of interacting floes).
It contains all geometric data like relative positions, velocities, 
intersection region (area, shape).
\\
\paragraph{Interposition vectors} of elements` centers and center of intersection region:
\begin{equation}
 \underline{\vec{r_{12,cd}}} = \hat{e2\_to\_e1} \cdot \vec{NORTH}
\end{equation}
\begin{equation}
 \underline{\vec{r_{1,cd}}} = \underline{\vec{CoI}}
 \end{equation}
\begin{equation}
 \underline{\vec{r_{2,cd}}} = \underline{\vec{r_{1,cd}}} - \underline{\vec{r_{2,cd}}}
 \end{equation}
 \\where $\hat{e2\_to\_e1}$ is a transformation matrix from $e_2$ local coordinates
 to $e_1$ local coordinates, $\vec{NORTH} \equiv (0, 0, 1)$,
 $\underline{\vec{CoI}}$ is a vector from $e_1$ center (coinsides with the origin)
 to the center of intersection region (calculated by ``geometry'' module).
 
\paragraph{Velocities}:
\begin{equation}
 \vec{v_{1,cd}} = \vec{V_{e_1}} + (\underline{\vec{r_{1,cd}}} \cdot R) \cdot \Omega_{z,e_1}
 \end{equation}
\begin{equation}
 \vec{v_{2,cd}} = \hat{e2\_to\_e1} \cdot \vec{V_{e_2}} + (\underline{\vec{r_{2,cd}}} \cdot R) \cdot \Omega_{z,e_2}
 \end{equation}
\begin{equation}
 \vec{v_{12,cd}} = \vec{v_{1,cd}} - \vec{v_{2,cd}}
\end{equation}
\\where $\Omega_{z,e_j}$ is an angular velocity of $e_j$ element about it`s local
vertical axis, $\vec{V_{e_j}}$ is translational velocity of $e_j$.

\subsection{Pairwise forces}

All pairwise forces take $contact\ data$ and other contact props 
as the parameters and generate small structure containing 
the vector of force, two points of applications of this force
to floes and the force couple value. In this section $N$ denotes 
combined torque from possible moment of $F$ and force couple caused 
by any kind of source.

Several methods which include elastic forces use the spring-like 
elasticity coefficient:
\begin{equation}
 El_{1\ 2} = \frac{El_1 \cdot El_2}{El_1 + El_2} =...= 
 \sigma \frac{h_1 \cdot h_2}{h_1 \underline{r_2} + h_2 \underline{r_1}} \cdot R^{-1}
\end{equation}
where $\sigma$ is stiffness of ice, $h_{1, 2}$ are thickness of the 
largest layers of two floes, $r_{1, 2}$ are the ``length'' of elements
(approximately the distance from the centers or ice elements to
intersection region).
\\
\\
There are next basic force methods:

\paragraph{Collision} - interaction of two separate floes.
\begin{equation}
 \vec{F_c} = El_{1\ 2} \cdot (\underline{A_{cd}} \cdot R^2) \cdot ``rigidity'' +
 \eta \cdot (-\vec{v_{1 2,cd}}) \cdot ( \underline{A_{cd}} \cdot R^2) \cdot ``viscosity''
\end{equation}
\begin{equation}
 N_c = [\vec{r} \times \vec{F_c}]
\end{equation}
where $\eta$ is parameter of ice-ice interaction viscosity, 
``rigidity'' and ``viscosity'' are manual factors for possible 
`switching-off' of unwanted forces, $r$ - application point of force.

\paragraph{Test\_spring} is the simpliest method for two frozen 
floes (Joint). Contact is represented as a single 1D spring with 
the same stiffness as if it was an ice block of appropriate size.
\begin{equation}
 \vec{F_t} = (\underline{\vec{\Delta r}} \cdot R) \cdot El_{1\ 2} \cdot (\underline{wid} \cdot R) \cdot dur 
\end{equation}
\begin{equation}
 N_t = [\vec{r} \times \vec{F_t}]
\end{equation}
where $\underline{\vec{\Delta r}}$ is the displacement from the 
equilibrium position (which [position] is defined at the moment of
Joint freezing), $\underline{wid}$ is width of contact (length of 
mutual edge of two polygons), $dur$ is current durability of Joint
(1 for new unbroken Joint, 0 for destructed one).

\paragraph{Distributed\_spring} is the main method for two frozen
floes. Contact is represented as 2D spring, that allowes to calculate
forces and moment far more precisely.
\begin{equation}
 \vec{F_d} = (\frac{\underline{\vec{\Delta r'}}+\underline{\vec{\Delta r''}}}{2} \cdot R) \cdot 
 El_{1\ 2} \cdot (\underline{wid} \cdot R) \cdot dur
\end{equation}
\begin{equation}
 N_d = [\vec{r} \times \vec{F_d}] + 
 [(\frac{\underline{\vec{\Delta r'}}-\underline{\vec{\Delta r''}}}{2} \cdot R) \times (\underline{\vec{wid}} \cdot R)] 
 \cdot El_{1\ 2} \cdot (\underline{wid} \cdot R) \cdot dur \cdot \frac{1}{12}
\end{equation}
where $\underline{\vec{\Delta r'}}$ and $\underline{\vec{\Delta r''}}$
are the displacements of two ends of mutual edge of polygons, 
$\underline{\vec{wid}}$ is a vector of that very edge itself.

\subsection{Joint state and props}

\paragraph{Durability} of each Joint evolves by the next rule:

 $if\ \frac{\Delta r_{max}}{len}\ >\ ``tensility''\ :\ dur\ -=
 \ \frac{<\Delta r>}{len} \cdot ``solidity'' \cdot dt$
 
\paragraph{Fastening} is not that simple.
For proper check of land fastening conditions at each time step
ice elements accumulate the area of overlaping ($OA_{e_j}$) by
border elements (aka static elements on seashores). In addition 
each ice element stores the area minimal area ($A_{min,e_j}$)
of single floe amoung the nearest floes (including self).
 
\begin{equation}
 OA_{e_j} = \sum_{c \in C} A_{intersect[e_j,c]} 
\end{equation}
\begin{equation}
 A_{min,e_j} = \min_{c \in C} [A_c]
\end{equation}
where $C$ is a set of nearest ice elements (including self), 
 
\section{Dynamics}
 
\begin{equation}
 SuperTorque = ( \frac{-F_y}{R \cdot m}, \frac{F_x}{R \cdot m}, \frac{N}{I})
\end{equation}
\begin{equation}
 e.W += SuperTorque \cdot dt
\end{equation}

\section{Position}

\paragraph{1) propagation} (axis is random Global):
%\right\{
\begin{eqnarray}
  \untqtrn{p} = (1, \frac{(W_x, W_y, 0) \cdot dt}{2}) \\
  \untqtrn{q} = [\untqtrn{q} \times \untqtrn{p}]
\end{eqnarray}

\paragraph{2) spin} (axis is Local $\vec{Oz}$):
%\right\{
\begin{eqnarray}
  \untqtrn{p} = (1, \frac{(0, 0, W_z) \cdot dt}{2}) \\
  \untqtrn{t} = [\untqtrn{q} \times \untqtrn{p}]
\end{eqnarray}

\paragraph{3)} transform old e.W and e.q according to element`s 
new position and orientation
%\right\{
\begin{eqnarray}
  W = loca\_to\_local( \untqtrn{t}, \untqtrn{q}, W ) \\
  \untqtrn{q} = normilized(\untqtrn{t})
\end{eqnarray}
where $loca\_to\_local$ transforms $W$ from $\untqtrn{q}$ to 
$\untqtrn{t}$ local coordinates
