% FILE: geometry2d.tex

\chapter{2D Geometry}\label{ch:geometry2d}

%----------------------------------------------------------------------
\section{Convex Polygon Intersection}

The method implements O'Rourke's approach from his
book~\cite{bib:orourke}. The notation and some details were updated to
make the method easy to describe.

Let $\bm{P}=\{P_1,P_2,\dots,P_N\}$ and $\bm{Q}=\{Q_1,Q_2,\dots,Q_M\}$
be two convex polygons. We need to find another convex polygon
$\bm{X}=\bm{P}\cap\bm{Q}$.

Initially we choose (arbitrary) a side $\bm{p}=[P_1,P_2]$ and a side
$\bm{q}=[Q_1,Q_2]$. We use indices $i$ and $j$ to indicate the
first vertex of $\bm{p}$ and $\bm{q}$ correspondingly. 

We say that point $X\in {\cal L}(\bm{p})$ if $X$ lies on the left side
(half-plane) of the directed segment $\bm{p}=[P_i,P_{i+1\mod N}]$. This
condition is easy to check as follows:
\begin{equation}
  X \in {\cal L}(\bm{p}) =
  \left[ {\cal H}\left( { \bm{p} \times (X-P_i) } \right) = 1 \right]
\end{equation}
where ${\cal H}$ is Heavyside function:
\begin{equation}
  {\cal H}(x) = \left\{
  \begin{aligned}
    1, &\quad \text{if}\,\,x > 0,\\
    0, &\quad  \text{if}\,\,x \le 0.\\
  \end{aligned}\right.
\end{equation}

We introduce a function $L(\bm{p},X) = \left[{\cal H}\left( { \bm{p} \times
  (X-P_i) } \right) = 1\right]$ for convenience.

O'Rourke derives the ``advance rule'', e.g. the rule which polygon to
choose next for advancing the side. The rule can be summarized as the
Table~\ref{tab:arule}.
%
\begin{table}
  \center
  \caption{O'Rourke's ``Advance rule''.}
  \begin{tabular}{c|c|c|c}
    \hline
    $\bm{p}\times\bm{q} > 0$ & Halfplane condition & Programming rule & Advance rule \\
    \hline
    true & $Q_j\in {\cal L}(\bm{p})$     & $L(\bm{p},Q_j)=\text{true}$ & P \\
    %
         & $Q_j\notin {\cal L}(\bm{p})$  & $L(\bm{p},Q_j)=\text{false}$ & Q \\
    \hline
    false & $P_i\in {\cal L}(\bm{q})$    & $L(\bm{q},P_i)=\text{true}$ & P \\
    %
          & $P_i\notin {\cal L}(\bm{q})$ & $L(\bm{q},P_i)=\text{false}$ & Q \\
    \hline
  \end{tabular}
  \label{tab:arule}
\end{table}

The algorithm
