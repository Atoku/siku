% FILE: geometry2d.tex

\chapter{2D Geometry}\label{ch:geometry2d}

%----------------------------------------------------------------------

\section{General info}

Due to different conditions and obstacles siku geometry is placed in
separate module. For better performance we avoid using any external
libraries (like boost) for two-dimensional geometry. 

All functions are made by hand (even if they just implement generally 
well-known algorithms).

2D geometry module contains it`s own set of tests for all provided
functions.

%----------------------------------------------------------------------
\section{Convex Polygon Intersection}

\subsection{Advance rule}

The method implements O'Rourke's approach from his
book~\cite{bib:orourke}. The notation and some details were updated to
make the method easy to describe.

Let $\bm{P}=\{P_0,P_2,\dots,P_{N-1}\}$ and
$\bm{Q}=\{Q_0,Q_2,\dots,Q_{M-1}\}$ be two convex polygons, vertices
ordered counter-clockwise. We need to find another convex polygon
$\bm{X}=\bm{P}\cap\bm{Q}$.

Initially we choose (arbitrary) a side $\bm{p}=[P_0,P_1]$ and a side
$\bm{q}=[Q_0,Q_1]$. We use indices $i$ and $j$ to indicate the
first vertex of $\bm{p}$ and $\bm{q}$ correspondingly. 
We say that point $X\in {\cal L}(\bm{p})$ if $X$ lies on the left side
(half-plane) of the directed segment
$\bm{p}=[P_i,P_{i+1}]$\footnote{Everywhere we assume $i+1$ is actually
$(i+1) \mod N$, and $j+1$ is actually $(j+1)\mod M$. We omit it here
  for briefty}. This
condition is easy to check as follows:
\begin{equation}
  X \in {\cal L}(\bm{p}) =
  \left[ \left( { \bm{p} \times (X-P_i) } \right) > 0 \right]
  = L ( \bm{p}, X ).
\end{equation}
Where we introduced a function $L(\bm{p},X) = \left[\left( { \bm{p}
    \times (X-P_i) } \right) > 0\right]$ for convenience.

O'Rourke derives the ``advance rule'', e.g. the rule which polygon to
choose next for advancing the side. The rule can be summarized as the
Table~\ref{tab:arule}.

The algorithm described in the table also uses a special $\chi$ inside
status. It may have 3 values: ``U'' (for unknown), ``P'' (for inside
polygon P), and ``Q'' (for inside polygon Q). The value of $\chi$
changes when an intersecton of two segments detected. It is set as
described also in Table~\ref{tab:arule}.  It does not change if both
conditions are not true.
%
\begin{table}
  \center
  \caption{O'Rourke's ``Advance rule''.}
  \begin{tabular}{c|c|c|c|c}
    \hline
    $\bm{p}\times\bm{q} > 0$ & Programming rule & Advance rule & Add
    vertex condition & Vertex to add\\
    \hline
    true  & $L(\bm{p},Q_{j+1})=\text{true}$ & P & $\chi=P$ & $P_{i+1}$ \\
    %
          & $L(\bm{p},Q_{j+1})=\text{false}$ & Q & $\chi=Q$ & $Q_{j+1}$ \\
    \hline
    false & $L(\bm{q},P_{i+1})=\text{true}$ & Q & $\chi=Q$& $Q_{j+1}$ \\
    %
          & $L(\bm{q},P_{i+1})=\text{false}$ & P & $\chi=P$ & $P_{i+1}$ \\
    \hline
  \end{tabular}
  On intersection only $\chi$-rule:
  \begin{tabular}{rll}
    $L(\bm{q},P_{i+1})=\text{true}$ & $\Rightarrow$ & $\chi=P$, \\
    $L(\bm{p},Q_{j+1})=\text{true}$ & $\Rightarrow$ & $\chi=Q$.
  \end{tabular}
  \label{tab:arule}
\end{table}

\subsection{Termination and continuation condition}

Let $i_x$ be the number of advances of $P$ since first intersection,
and $j_x$ be the number of advances of $Q$ since first
intersection. In this case, the loop continuation condition is
\begin{equation}
  \left( [i_x < N] \lor [j_x < M] \right) \land [i_x < 2N] \land [i_y < 2M].
\end{equation}


