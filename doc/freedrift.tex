%%% File: freedrift.tex

\chapter{Drift testing}

%----------------------------------------------------------------------

\section{Introduction}

It is obvious that our program should contain basic tests with easily 
observable results. These tests can either confirm that all calculations
are correct or show up that there are some miscalculations or phisical
laws which are not taken into account.

For proper testing we provide several simple tests, which could be ran
directly from 'freedrift.py', and some more complicated hardcoded tests, 
which require rebuilding the C++ core, that should not be made by users. 
More detailed description will be given below.

%----------------------------------------------------------------------

\section{User-restricted tests}

\subsection{Hardcoded winds}

In order to remove any possibilities to make a mistake while passing 
wind data from file to python and from python to core - a simple windfield 
has been hardcoded into C++ modules.

That field gives wind velocity by default equal to (-10, 1) in convention 
(east-velo, north-velo) in current velocity units all around the globe.
The values of components can be changed in module 'vecfield.cc' in method
'get\_at\_lat\_lon\_rad. This test can be turned on in module 'globals.hh':
one should uncomment testing constructor (next to atmospheric wind data 
declaration), and by default is turned off.

As the result - all elements on the globe should behave according to 
hardcoded wind. \textbf{NOTICE: hardcoded wind will not be outlined as
it is not connected with python!}

\subsection{Hardcoded spins}

In early stages, while interaction between elements is not properly 
defined, the most basic spin test has been hardcoded. The test is naive 
constant angular velocity of all elements. This velocity is brutally 
embedded inside 'dynamics.cc' module. The values are chosen for different
elements` behavior (and equal to zero by default). 

When angular velocity is small - elements behave like 'leaves on the wind',
simply drifting in direction of wind velocity, but slower in the reason 
of water. 
In the case of medium angular velocity - elements begin to turn aside from
wind direction while they are mowing. This effect is caused by some 
kind of inner model gyro forces.
When angular velocity is too high - elements behave more like spinning top:
they are moving slowly with fluctuations or not moving at all.
\textbf{NOTICE: angular velocity and as the result elements behavior 
depends on time scaling of current computing session!}

%----------------------------------------------------------------------

\section{Generic python tests}

\subsection{Output settings}

Befor starting the test itself, one should set a proper output for results.
In order to observe the results we provide besic GMT picturing called from 
python modules.

All preruntime picture settings are located in 'plot\_config.py' file and 
are optional.Most values have several samples and described below:
\begin{itemize}
\item 'uwind\_file' and 'vwind\_file' - string that contains the name of 
 .nc files with uwind and vwind variables. 'gmt\_plotter.py' automaticly
 prints wind grid from that files.
\item time\_index - integer, that specifies the time from .nc files for
 selecting the wind grid.
 \textbf{Totally optional: runtime setting provided}
\item 'grid\_step\_lat' and 'grid\_step\_lon' - grid steps in .nc files for 
 proper interpoaltion and plotting.
\item 'out\_pic\_name' - name of file for GMT output.
 \textbf{Totally optional: runtime setting provided}
\item 'verbose' - verbosity state
\item 'inter\_domain' - the domain for gmt\_plotter inner interpolation.
 Is being given in convention (minlon, maxlon, minlat, maxlat) in degrees.
 Several main domains are provided and commented out.
\item 'inter\_density' - average density of additional interpolated vectors
 in degrees. 0 - for canceling additional interpolation.
\item 'view' - \textbf{Important!} main option! Contains information about
 map projection, oriantation and other minor gmt settings.
 Various views are provided and commented out.
\item 'ground\_colr' - color of ground.
\item 'coasts' - \textbf{Important!} second main option, but generally
 optional as it has good dafault value.
 Contains gmt command string for printing coast lines.
\item 'vector\_scaling' - additional scaling for wind vectors.
 For most maps dafeult value suits well, but when entire globe is being 
 printed - vectors can cover all other output, so should be resized 
 manually.
\item 'inter\_wind' - gmt command line for printing interpolated winds. File
 name is synchronized with interpolation module.
\item 'grid\_wind' - gmt command line for printing grid winds. File
 name is synchronized with interpolation module.
\item 'underlays' and 'overlays' - lists of optional underlays and overlays
 (accordingly), such as monitoring output, additional grids or objects.
\end{itemize}

\subsection{Testing scenarios}

Same to all other scenarios - these are being set inside user scenario 
python files that are given as an argument to core binary (siku) on 
execution.

Current testing scenario settings are being contained in 'freedrift.py', 
and are described below.

\subsection{Materials}

Each element has various properties. A lot of them can be simply described
the material, which element is made of. 

Default material for model is ice.  Material`s parameters can be changed in 
'material.py'. Inside the scenario file all materials are being loaded into 
table in 'Define material' section.

\subsection{Wind initializations}

Befor anything can move one should define a wind grid for inner processing.
Yet only NMC reanalysis data is supported. 

Grid loading ang preprocessing examples are being located in 
'Wind initializations' section of 'freedrift.py'.
First of all one should load siku.wind field as a proper NMCSurfaceVField
object, that depends on two NMCVar objects for two wind components 
(described in 'wnd,py' and 'nmc.py' accordingly).
After the loading one can additionaly make some changes in that wind grid 
(such as setting all values equal to test values by 'make\_test\_field' 
method. It also can be applied after each loading or updating wind grid).

The 'make\_test\_field' method simply sets all values of wind velocity equal 
to that, which are given as arguments (in terms (east-velo, north-velo)).
If arguments are empty - the winds all over the glob will be set as huge 
cyclones. That is very usefull for checking global drift and spin either 
with or without detection land collisions. 

\subsection{Date/time}

As the model describes the dynamics - another inevitable setting is time.

Several time examples are being located in 'date/time settings' section 
in 'freedrift.py'. Generally time should be set by exact values, but
as every NMC file contains it`s own set of times, one can easily choose 
some elements from that set.

There are several time variables, that should be set:
\begin{itemize}
 \item siku.time.start - starting time of modelling
 \item siku.time.finish - finish time (modelling shall end after reaching 
 that time)
 \item siku.time.last - the time of last calculation step
 \item siku.time.last\_update - the time of last wind update (used by 
 wind updating methods)
 \item siku.time.dts - outer cumputation time limit. Program shall exit 
 after it would have been working for that amount of time.
 \item siku.time.dt - \textbf{Important!} time step of the model. This 
 value has huge influence on computing process (mostly by dynamics).
\end{itemize}

\subsection{Poligons}

All elements are being represented as geometric poligons with different 
properties. Most general methods (such as loading) are being provided by 
'Polygon' class. The object of that class is being initialized in 
'Polygon initialization' siction.

\subsection{Elements}

The most basic example of several simple elements is being located in 
'elements' section. 

Firstly one shoud set a list of vertices for each element.
Secondly one have to init each element as an object of 'Element' class
using 'Polygon' object, matching material and set of thicknesses specifying 
monitoring functions.
In the third place one should append these elements to the list of elements.

\subsection{Monitoring function}

The variables and methods in section 'Monitor function for the polygon'
are being used for monitoring (such as picturing) some (or all) of the 
elements. 

Currently there are next objects:
\begin{itemize}
 \item siku.plotter - an object of GMT\_Plotter class for picturing frames
 with GMT.
 \item diagnostics.monitor\_freq - testing variable, defines frequency
 (actually period) of making frames in units of timestem, thus it is integer.
 \item siki.drift\_monitor - monitor function for elements.
 \item siku.diagnostics.step\_count - counter of diagnostics steps.
\end{itemize}

\subsection{Diagnostics}

The vatiables and methods for diagnostics are being located in 'Diagnostics 
function for the winds' section.

\textbf{Currently not used}

\subsection{Callback}

All python methods are being called from C++ core as a call-back functions.
Therefore this methods has been declared in 'Callback flag-mask generator'
section and defined afterward.

These functions are:
\begin{itemize}
 \item siku.callback.pretimestep - functions, that is called befor main 
 camputations. Contains different preprocessings, such as wind updating.
 
 Currently updating is turned off because time step is too large and couse 
 unwanted updates spoiling results.
 \item siku.callback.aftertimestep - function, that is called after main
 computations. 
 
 Currently contains picturing of elements and winds.
 \item conclusions - function, that is called only once - befor program 
 exits. 
 
 Currently contains command for creating a .gif animation from all frames.
\end{itemize}

\subsection{Current testings}

Current version of freedrift.py contains scenario for testing global 
movement and spinning. Running the model with that scenario should make
255 time steps covering 55 days, create 51 .eps file with the locations of 
three poligons and finally make .gif animationt of their movement. 

Expected work time is about 30-60 seconds (mostly spent for picturing).

For drift testing with real winds one should comment out last 3 lines in 
'Wind initializations' section in 'freedrift.py'. That will set testing 
wind grid with povided .nc files.

For changing projection one should comment out current 'view' variable 
in 'plot\_config.py' and uncomment that view, which one wants to see 
(or define his own).

%----------------------------------------------------------------------
