%%% File: mesh.tex

\chapter{Random points generation}


%----------------------------------------------------------------------

\section{Problem Statement}

We need to generate a set of points on sphere that will serve the role
of nuclei for Voronoi tesselation procedure. The set needs to satisfy
the following conditions:
\begin{enumerate}
\item
  The point set need to be randomly generated with some distribution
  (resolution) function over sphere. This means that some areas will
  have finer resolution than others.
\item
  The point set need to have blue noise characteristics. This means
  that the minimum distance between two points cannot exceed some
  limit value $d_{\min}$. This limit value is also a function of
  location.
\item
  The boundaries need to be also correctly represented. This means
  that the set of points need to contain explicitly the points from
  the boundaries. The resolution near the boundaries should be set by
  resolution function.
\end{enumerate}

The problem is not trivial. The standard approaches for generation of
blue noise are not always applicable here. On the other hand, we do
not need too sophisticated approach as the statistical quality of the
final set is not that critical.

The points are generated with \texttt{mesh/gen\_points.py} script that
is maintained to satisfy the procedure described in this chapter.

%----------------------------------------------------------------------

\section{Random points on sphere}


\subsection{Uniform distribution on sphere}\label{sec:marsaglia}

To generate a uniform point distribution a trivial random number
generation in the cube $(x,y,z)$ does not work. To pick a random point
on sphere we use the method by~\cite{bib:marsaglia1972}. The random
point $(x,y,z)$ on a sphere is generated using the following rule:
\begin{enumerate}
\item Generate X and Y, uniformly distributed on a segment (-1,1).
\item Reject all the points that $X^2+Y^2\ge 1$.
\item Generate a point on a sphere from the remaining points as
  follows:
\begin{equation}
  \begin{aligned}
    x &=& 2X\sqrt{1 - X^2 - Y^2},\\
    y &=& 2Y\sqrt{1 - X^2 - Y^2},\\
    z &=& 1 - 2(X^2+Y^2)
  \end{aligned}
\end{equation}
\end{enumerate}

\subsection{Blue noise}\label{sec:bluenoise}

The blue noise point generation is when the points are almost randomly
generated but with additional condition that there is no two points
that are closer than a particular distance $d_{\min}$. The simplest
way to generate blue noise is to remove a subset of the points from
the original set such that a new set will satisfy the blue noise
condtion. We proposed the following algorithm for this.

\begin{enumerate}
\item
  Sort all the points by $x$ coordinates (we will compare the
  projection of the points on $x$-axis).
\item
  Form a new empty list that will contain the blue noise set. Add the
  first point from the original list to it.
\item
  Traverse the original list starting from its second point. For each
  original point check the points from the new list backwards until
  the projection distance is smaller than $d_{\min}$. If no points are
  actually closer than $d_{\min}$, add original point to the new
  list. Otherwise, skip it.
\end{enumerate}

The new list will obviously satisfy the blue noise condition.

\subsection{Large voids}

There is no garantee that some points will be too isolated from the
set with much voids around. This problem can be solved by adding
points in the voids. However, we will not address it for now.

\subsection{Changing the distribution}

However, we need to generate the point set with varying distribution
function.

The first and simplest approach would be to generate the point set
with the finest resolution possible. Then we can apply the blue noise
filter technique described in Sec.~\ref{sec:bluenoise}. However, this
can be extremely slow even for not so fine resolution.

Instead, we can set the areas with different resolutions and the
simplest way is to generate uniform resolution in that areas and then
apply filter to the whole globe.

\subsection{Uniform distribution is selected region}

The method from Sec.~\ref{sec:marsaglia} cannot be directly
apply. However, we can use different approach. Technically, the
uniform distribution over the globe can be generated by
%
\begin{equation}
  \theta = 2\pi u, \quad \phi=\arccos( 2v-1 ),
\end{equation}
%
where $u$ and $v$ are random values on $(0,1)$.
