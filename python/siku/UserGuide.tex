\documentclass[a4paper,10pt]{article}
%\documentclass[a4paper,10pt]{scrartcl}

\usepackage[utf8]{inputenc}

\title{GMT\_Plotter User Guide}
\author{Botangens}
\date{1.08.2015}

\pdfinfo{%
  /Title    (GMT\_Plotter User Guide)
  /Author   (Botangens)
  /Creator  ()
  /Producer ()
  /Subject  ()
  /Keywords ()
}

\begin{document}
\maketitle

%\begin{center}
 %{\huge GMT\_Plotter User Guide \normalsize}
%\end{center}

%------------------------------------------

\section{General information}

GMT\_Plotter is python interface class for 'siku' project.

It uses several other python classes for reading wind input
data and configuration settings, constructing check-positions 
grid, interpolation of wind velosity and actually drawing .eps file.

%---------------------------------------

\section{Config file}

'plot\_config.py' is cinfiguration file for GMT\_Plotter class.

It contains various settings and is used at the begining of plotting procedure.


Although all variables are optional - user MUST specify interpolation domain
and drawing view.

\subsection{inter\_domain}

'inter\_domain' is a tuple that contains info about boarders 
of interpolation region in next format:
( min\_lon, max\_lon, min\_lat, max\_lat )

\subsection{view}

'view' is string of characters that spacify output region, projection 
and coordinates grid parameters.

\begin{itemize}
 \item '-G' is output region in format:
 -Gmin\_lon/max\_lon/min\_lat/max\_lat
 
 NO SPACES ALLOWED!
 
 \item '-J' is projection specification. For more read GMT official 
 guide or use predefined values
 
 \item '-B' is coordinates grid parameters. For more read GMT official 
 guide or just write '-BagX', where X - is resolution in degrees
\end{itemize}

\subsection{scale}

Other VERY important value is 'vector\_scaling'.
It`s a factor for correct vector size on output. 

Generaly Vectors are being scaled automaticly, but in some projections 
hardcoded parameters become too small for proper vectors output.
If that happens - user could (should) scale the resulting vectors 
at his own taste.

\end{document}
